\subsection{Apology}
The sequel of at least two years working with the IT cabinet's current network is a list of shortages:
\begin{enumerate}
 \item The network's reliability is nothing being granted,
 \item The network's common data storage provides no kind of security but everlasting disappearance of data due to user action,
 \item The network's technology in software aspects has years ago become timeworn,
 \item The network's security circumstances are in a state worse than outdated
 \item And the network's contemporary infrastructure in both hard- and software aspect restrains its users in doing adequate and fair work.
\end{enumerate}
Therefore, the signing people concern the existing infrastructure as of not a fraction of the benefit for any tutorial or lesson which might be possible with modern technologies in hardware and especially open source software and have arranged themselves in order to pass in suggestions of improvement as they are listed:
\begin{enumerate}
 \item The software infrastructure of at least one of the IT cabinet's rooms wil be replaced by a GNU/Linux based one, including server and client, providing a solution far more cost-efficient than the existing one,
 \item The improvement in secureness will be achieved using the widely acknowledged technologies provided by both the facilities of the operating system being introduced and appropriate software being used. The latter will be mostly software licensed with some free software license and deligently chosen -- hence there will be always someone fathering upon the used product.
 \item The data storage method will also be reengineered for not being everlasting but inexpensive to maintain (regarding cost and time consumption) and always backed up.
 \item Along with the storage method, the rights management has to change, too. Every user is supposed to have his own user account and a private desktop environment providing a data storage which is secured against excessive inflation. Every user's activities will be jailed into his personal configured environment and the load an user generates (network bandwith and disk storage he consumes) will underly a steadily reconfigured restriction policy preventing him from interfering with his classmates' needs.
 \item Having the previous points realized, there will be an opening in lesson and tutorial processing: first, the new environment provides better, various software and also new ways of working; due special activity restrictions, pupils will be able to experience otherwise secure perverting features on their local machine in real -- e.g. \emph{iptables}, \emph{ACLs} -- being nevertheless essential to both system and network programming and administration. We will implement a fine distribution of permissions, thus enabling every person requiring it (e.g. teachers) to perform necessary management tasks (including the granting of rights).
\end{enumerate}

For the given reasons, it is necessary to break with the contemporary system to keep the work environment of all people involved as productive as possible. Regarding on financial aspects, the following shall not remain unmentioned:
\begin{itemize}
 \item The software being designated to be used comes for free; free means not only "free of cost" like in "free beer" but also "free of restrictions" like in "freedom of speech".
 \item All client PC hardware is going to be used as-is, apart from restraining guardian cards simply going to be removed because they are limited to work with Microsoft Windows and counterfeiting any attempts of enhancing system security by installing necessary updates to the utmost extent;
 \item The new server will be built from low-cost yet reliable consumer parts, combined in a way making the system failsafe \emph{and} cost-efficient (for a school like ours). The cost being additional in acquisition compared to an existing machine will pay off in energy bills of a fifth of the current value. The new system will consume less than 100W (yes, one houndred Watts) even under peak load. Additionally the new server will be much more powerful than the old hardware, providing at least thrice the processing power, ten up to twenty times the network bandwith and ten times the data storage, embellishing the user experience to a value worth the year 2010. We estimate the cost for the new hardware (please note: there is no money needing to be paid for software) at 550\euro. Assuming the new system will consume 100W and run as long as the old system which consumes 200W (idle, all these values are conservative estimations), assuming an energy cost of 20ct/kWh while the system is running 300 days 24 hours a day (as we saw this is the current state) it would save bills over 144\euro every year, in real this value will be bigger because we can tweak a lot of things, for example we could shut down our low-power solution overnight, up on that modern processor architectures and chipsets support a partial shutdown of the system in order to save a lot of Watts.
\end{itemize}

 
\subsection{LCP team}

\begin{tabular}{|l|l|c|}
 \bfseries{Name} 	& \bfseries{E-Mail address} 		& \bfseries{$\infty$}	\\
\hline

Konrad Brust 		& \nolinkurl{<>} 			& 11N -- 734 		\\
Jan Sebastian Götte 	& \nolinkurl{<jsgoette@yahoo.de>} 			& 11N -- 720 		\\
Robert Hemstedt 	& \nolinkurl{<>} 			& 10N -- 351 		\\
Jens Oliver John 	& \nolinkurl{jens.o.john@gmail.com} 	& 11N -- 720 		\\
Hr. Uwe Krägefski 	& \nolinkurl{<>} 			& Projektleiter 	\\
Igor Lückel 		& \nolinkurl{<>} 			& 11N -- 734

\end{tabular}
